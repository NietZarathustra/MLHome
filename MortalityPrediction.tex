\documentclass[a4paper,12pt]{article}

% Sprachpaket für Deutsch
\usepackage[utf8]{inputenc}
\usepackage[ngerman]{babel}

% Für PDF-Bilder
\usepackage{graphicx}

% Paket für Referenzen
\usepackage{hyperref}

% Titel, Autor und Datum
\title{Brustkrebs Indicatoren}
\author{Timo Michaelis}
\date{\today}

\begin{document}

% Titelbild
\maketitle

\begin{abstract}
Im Folgenden ist das Ziel ein Vorhersage über die Mortalitätsrate von Krankheiten im Rahmen der Parameter, auf welche
    ein Staat direkten Einfluss haben kann

\end{abstract}

\tableofcontents % Inhaltsverzeichnis

\newpage

\section{Einleitung}
Ziel dieser Ausarbeitung ist die Unterstützung zur effizienten Mortalitätsverringerung bei Krankheiten/Pandemien, dafür
wird eine \textit{Prediction} erstellt, in Abhängigkeit von den folgenden Indikatoren.\newline
Es wird also eine Vorhersage getroffen wie groß die Mortalitätsrate von beliebigen Pandemien ausfällt in Abhängigkeit des
Status eines Staates. Dafür wird die Datengrundlage von .... genutzt, damit das Modell trainiert werden kann.
Indes dient diese Ausarbeitung lediglich als Grundlage, da hier nur die Prediction aufgebaut wird. Die Folgearbeit
behandelt das Optimierungsproblem aufgrundlage der Prediction und den verfügbaren Mitteln eines Staates.\newline
Dennoch bietet die Prediction einen Einblick in die momentane Erwartungshaltung einer Pandemie

\section{Datensatz}
Der verwendete Datensatz stammt aus der \textit{Global Health Statisitic}, sie besitz eine Public Domain Lizenz und wird jährlich updated\newline
Folgende Daten sind enthalten
\begin{itemize}

\item Country: The name of the country where the health data was recorded.

\item Year: The year in which the data was collected.

\item Disease Name: The name of the disease or health condition tracked.

\item Disease Category: The category of the disease (e.g., Infectious, Non-Communicable).

\item Prevalence Rate (%): The percentage of the population affected by the disease.

\item Incidence Rate (%): The percentage of new or newly diagnosed cases.

\item Mortality Rate (%): The percentage of the affected population that dies from the disease.

\item Age Group: The age range most affected by the disease.

\item Gender: The gender(s) affected by the disease (Male, Female, Both).

\item Population Affected: The total number of individuals affected by the disease.

\item Healthcare Access (\%): The percentage of the population with access to healthcare.

\item Doctors per 1000: The number of doctors per 1000 people.

\item Hospital Beds per 1000: The number of hospital beds available per 1000 people.

\item Treatment Type: The primary treatment method for the disease (e.g., Medication, Surgery).

\item Average Treatment Cost (USD): The average cost of treating the disease in USD.

\item Availability of Vaccines/Treatment: Whether vaccines or treatments are available.

\item Recovery Rate (\%): The percentage of people who recover from the disease.

\item DALYs: Disability-Adjusted Life Years, a measure of disease burden.

\item Improvement in 5 Years (\%): The improvement in disease outcomes over the last five years.

\item Per Capita Income (USD): The average income per person in the country.

\item Education Index: The average level of education in the country.

\item Urbanization Rate (\%): The percentage of the population living in urban areas.

\end{itemize}


\section{Methoden}
Als Methoden zur Erstellung des Vorhersagen Algorithmus habe ich zuerst \textit{ RandomForestRegression} benutzt, anschließend nutzte ich
\textit{SVR (Support Vector Regression)}.
\subsection{Random Forest Regression}
\subsection{Support Vector Regression}
\section{Ergebnisse}
Folgende Ergebnisse ergaben die Predictions
\subsection{Random Forest Regression}
\subsection{Support Vector Regression}
\section{Diskussion}
In der Diskussion analysierst du die Ergebnisse und ziehst Schlussfolgerungen.

\section{Fazit}
Mittels dieser Prediction lässt sich die Bereitschaft auf bestimmte Krankheiten bestimmen, ein weiterer Aspekt wäre die
Spezifierungsmöglichkeit der Krankheit oder der zu betrachtenden Probanten, hierzu
\newpage

% Anhang für das Jupyter Notebook
\appendix
\section{Anhang: Jupyter Notebook}
Hier kannst du dein Jupyter Notebook einfügen. Eine Möglichkeit, ein Jupyter Notebook als Anhang einzufügen, ist die Umwandlung des Notebooks in ein PDF oder ein anderes unterstütztes Format und das Hinzufügen als Bild oder Dokument.

% Beispiel für ein eingefügtes PDF des Jupyter Notebooks
\begin{figure}[h!]
    \centering
    \includegraphics[width=\textwidth]{notebook.pdf}
    \caption{Jupyter Notebook als Anhang}
\end{figure}

\end{document}
