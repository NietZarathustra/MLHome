\documentclass[a4paper,12pt]{article}

% Sprachpaket für Deutsch
\usepackage[utf8]{inputenc}
\usepackage[ngerman]{babel}

% Für PDF-Bilder
\usepackage{graphicx}

% Paket für Referenzen
\usepackage{hyperref}
\usepackage{biblatex}

% Titel, Autor und Datum
\title{Brustkrebs Indicatoren}
\author{Timo Michaelis}
\date{\today}

\begin{document}

% Titelbild
\maketitle

\begin{abstract}
Im Folgenden wird versucht anhand verschiedener Indikatoren zur prognostizieren, ob eine Patientin Brustkrebs besitzt.
    Dieses Model dient lediglich der Hilfestellung, nicht aber dem Ersatz eines fachkundigen Arztes
\end{abstract}

\tableofcontents % Inhaltsverzeichnis

\newpage

\section{Einleitung}
In der vorliegenden Arbeit wird eine Unterstützung zur Erkennung von Brustkrebs geboten. Dafür werden verschiedene Eigenschaft
u.a. \textit{durchschnittlicher Radius, durchschnittliche Textur, usw.} berücksichtigt und eine wahrscheinliche Prognose geboten.\newline
Das Ziel dieser Arbeit ist es somit ein Modell zu erstellen, welches bei Input der benötigten Parameter eine vorläufige Diagnose stellt,
welche etwaig dem Patienten bzw. dem Arzt zu einer genaueren Untersuchen verleiten. Dies verringert somit nicht nur die die Chance von
sogenannten \textit{false positives}, sondern kann auch die Bereitschaft unterstützen häufiger notwendige Untersuchungen
durchzuführen, da die Untersuchung weniger komplex wird.\newline
%Literaturarbeit
Im folgenden wird grundsätzlich der Ursprung und Inhalt des Datensatzes diskutiert, sowie die darauf angewandten Methoden näher erläutert.




\section{Datensatz}
Der verwendete Datensatz stammt aus der \textit{Diagnostic Wisconsin Breast Cancer Database.}\cite*{sth}, wie erwartet befasst sich dieser
Datensatz mit Gesundheitlichen Aspekten und besitzt indes folgende \textit{Features}
\begin{itemize}
    \item mean radius
    \item mean texture
    \item mean perimeter
    \item mean area
    \item mean smoothness
    \item mean compactness
    \item mean concavity
    \item mean concave points
    \item mean symmetry
    \item mean fractal dimension
    \item radius error
    \item texture error
    \item perimeter error
    \item area error
    \item smoothness error
    \item compactness error
    \item concavity error
    \item concave points error
    \item symmetry error
    \item fractal dimension error
    \item worst radius
    \item worst texture
    \item worst perimeter
    \item worst area
    \item worst smoothness
    \item worst compactness
    \item worst concavity
    \item worst concave points
    \item worst symmetry
    \item worst fractal dimension
\end{itemize}
Hinzu kommt noch der \textit{Target-Datensatz}, welcher beschreibt ob der Datenpunkt jeweils \textit{Malignand} oder \textit{Benign}
ist.\newline
Bevor näher auf den Datensatz und seine zusammenhänge eingegangen wird, gilt es zu beurteilen inweit aufbereitet dieser ist.
Hiefür sei relevant zu testen ob teile nicht aufgefüllt sind, und man gegebenenfalls Teile löschen muss. Desweiteren sei noch der Typus
der Daten Konsistent und es sei zu überrpüfen, ob es unsinnige Einträge, wie einen negativen Radius gibt.
Dies wurde überprüft und der Datensatz ist vorbereitet.\newline
Desweiteren sei noch zu überprüfen wie viele Ausreißer dieser Datensatz besitzt und ob diese gegebenenfalls zu entfernen seien.
Dafür wurden mehrere Boxplot-Graphen erstellt, welche zwar Ausreißer angeben, diese sind physikalisch möglich und indes nicht ohne
beeinträchtigungen (underfitting) zu entfernen
\newline
%images boxplot und code?
Zur Frage ob es Abhängigkeiten zwischen den Features und dem Target gibt, lässt sich dazu eine Grundidee fassen, indem man verschiedene
Features im zwei Dimensionalen gegeneinander aufträgt:\newline
%image graph
Ebenfalls lässt sich mittels von Korrellationstabellen ebenfalls Korrellationen nachweisen, weswegen nun nur noch die geignete Methode
zu Erstellung eines Prediction-Algorithmus benötigt wird
\newline

\section{Methoden}
Es existieren zwei Kategorien in welche unsere Features einsortiert werden, insofern seien Algorithmen notwendig welche eine Klassifikation durchführen.
Für kleinere Datensätze, wie hier vorliegend, ist \textit{ RandomForestClassifier} gut zu nutzen und da bereits aus Graph
eine perfekte lineare separierung nicht möglich ist, ist \textit{SVM (Support Vector Machine)} ein passender Algorithmus.
\subsection{Random Forest Classifier}
Der Random Forest Classifier nutzt eine gewisse Anzahl an verschiedenen Entscheidungsbäumen mit dem Ziel die korrekte Entscheidung zu erfüllen,
dies führt dazu, dass die Nachteile eines Entscheidungsbaumes verringert (wie z.B. \textit{overfitting})
\subsection{Support Vector Machine}
Ein SVM nutzt die Tatsache aus, dass selbst wenn eine Separierbarkeit z.B. im 2-Dimensionalen nicht machbar ist,
diese in höheren Dimensionen durchaus möglich ist.
\subsubsection{Kernel}

\section{Ergebnisse}
Um das bestmöglichste Ergebniss zu erzielen, wurde ein Grid angewandt damit die bessere Anzahl an Entscheidungsbäumen bzw.
des Wertes der Fehlklassifizierungsstrafe genutzt werden kann.
\subsection{Random Forest Classifier}

\subsection{Support Vector Machine}
\section{Diskussion}
In der Diskussion analysierst du die Ergebnisse und ziehst Schlussfolgerungen.

\section{Fazit}
Mittels dieser Prediction lässt sich die Bereitschaft auf bestimmte Krankheiten bestimmen, ein weiterer Aspekt wäre die
Spezifierungsmöglichkeit der Krankheit oder der zu betrachtenden Probanten, hierzu
\newpage

% Anhang für das Jupyter Notebook
\appendix
\section{Anhang: Jupyter Notebook}
Hier kannst du dein Jupyter Notebook einfügen. Eine Möglichkeit, ein Jupyter Notebook als Anhang einzufügen, ist die Umwandlung des Notebooks in ein PDF oder ein anderes unterstütztes Format und das Hinzufügen als Bild oder Dokument.

% Beispiel für ein eingefügtes PDF des Jupyter Notebooks
\begin{figure}[h!]
    \centering
    \includegraphics[width=\textwidth]{notebook.pdf}
    \caption{Jupyter Notebook als Anhang}
\end{figure}

\end{document}
